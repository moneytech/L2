% produced using Pandoc - strongly recommended
\PassOptionsToPackage{unicode=true}{hyperref} % options for packages loaded elsewhere
\PassOptionsToPackage{hyphens}{url}
%
\documentclass[]{article}
\usepackage{fullpage}
\usepackage{lmodern}
\usepackage{amssymb,amsmath}
\usepackage{ifxetex,ifluatex}
\usepackage{fixltx2e} % provides \textsubscript
\ifnum 0\ifxetex 1\fi\ifluatex 1\fi=0 % if pdftex
  \usepackage[T1]{fontenc}
  \usepackage[utf8]{inputenc}
  \usepackage{textcomp} % provides euro and other symbols
\else % if luatex or xelatex
  \usepackage{unicode-math}
  \defaultfontfeatures{Ligatures=TeX,Scale=MatchLowercase}
\fi
% use upquote if available, for straight quotes in verbatim environments
\IfFileExists{upquote.sty}{\usepackage{upquote}}{}
% use microtype if available
\IfFileExists{microtype.sty}{%
\usepackage[]{microtype}
\UseMicrotypeSet[protrusion]{basicmath} % disable protrusion for tt fonts
}{}
\IfFileExists{parskip.sty}{%
\usepackage{parskip}
}{% else
\setlength{\parindent}{0pt}
\setlength{\parskip}{6pt plus 2pt minus 1pt}
}
\usepackage{hyperref}
\hypersetup{
            pdfborder={0 0 0},
            breaklinks=true}
\urlstyle{same}  % don't use monospace font for urls
\usepackage{longtable,booktabs}
% Fix footnotes in tables (requires footnote package)
\IfFileExists{footnote.sty}{\usepackage{footnote}\makesavenoteenv{longtable}}{}
\setlength{\emergencystretch}{3em}  % prevent overfull lines
\providecommand{\tightlist}{%
  \setlength{\itemsep}{0pt}\setlength{\parskip}{0pt}}
\setcounter{secnumdepth}{0}
% Redefines (sub)paragraphs to behave more like sections
\ifx\paragraph\undefined\else
\let\oldparagraph\paragraph
\renewcommand{\paragraph}[1]{\oldparagraph{#1}\mbox{}}
\fi
\ifx\subparagraph\undefined\else
\let\oldsubparagraph\subparagraph
\renewcommand{\subparagraph}[1]{\oldsubparagraph{#1}\mbox{}}
\fi

% set default figure placement to htbp
\makeatletter
\def\fps@figure{htbp}
\makeatother


\date{}

\begin{document}

\hypertarget{l2}{%
\section{L2}\label{l2}}

L2 is an attempt to find the smallest most distilled programming
language equivalent to C. The goal is to turn as much of C's
\protect\hyperlink{switch-expression}{control structures}, statements,
\protect\hyperlink{strings}{literals}, \protect\hyperlink{fields}{data
structure constructs}, and functions requiring compiler assistance
(setjmp, longjmp, \protect\hyperlink{examples}{alloca},
\protect\hyperlink{assume}{assume}, ...) into things definable inside L2
(with perhaps a little assembly). The language does not surject to all
of C, its most glaring omission being that of a type-system. However, I
hope the \protect\hyperlink{examplesreductions}{examples below} will
convince you that the result is still
\protect\hyperlink{closures}{pretty interesting}. And if that is not
enough, I recommend that you take a look at \href{src/compile.l2}{the
implementation of a self-hosting compiler for L2 that accompanies this
project} and compare it to \href{bootstrap/compile.c}{the compiler for
bootstrapping it written in C}.

The approach taken to achieve this has been to make C's features more
composable, more multipurpose, and, at least on one occasion, add a new
feature so that a whole group of distinct features could be dropped. In
particular, the most striking changes are that C's:

\begin{enumerate}
\def\labelenumi{\arabic{enumi}.}
\tightlist
\item
  irregular syntax is replaced by
  \protect\hyperlink{internal-representation}{S-expressions}; because
  simple syntax composes well with a non-trivial preprocessor (and
  \protect\hyperlink{expression}{no, I have not merely transplanted
  Common Lisp's macros into C})
\item
  loop constructs are replaced with what I could only describe as
  \protect\hyperlink{with}{a more structured variant of setjmp and
  longjmp without stack destruction} (and
  \protect\hyperlink{an-optimization}{no, there is no performance
  overhead associated with this})
\end{enumerate}

There are \protect\hyperlink{expressions}{10 language primitives} and
for each one of them I describe their syntax, what exactly they do in
English, the i386 assembly they translate into, and an example usage of
them. Following this comes a listing of L2's syntactic sugar. Then comes
a brief description of \protect\hyperlink{internal-representation}{L2's
internal representation and the 6 functions that manipulate it}. After
that comes a description of how \protect\hyperlink{meta}{a
meta-expression} is compiled. The above descriptions take about 8 pages
and are essentially a complete description of L2. Then at the end there
is a \protect\hyperlink{examplesreductions}{list of reductions} that
shows how some of C's constructs can be defined in terms of L2. Here, I
have also demonstrated \protect\hyperlink{closures}{closures} to hint at
how more exotic things like coroutines and generators are possible using
L2's \protect\hyperlink{jump}{continuations}.

\hypertarget{contents}{%
\subsubsection{Contents}\label{contents}}

\begin{longtable}[]{@{}lll@{}}
\toprule
\textbf{\protect\hyperlink{getting-started}{Getting Started}} &
\protect\hyperlink{expressions}{Expressions} &
\protect\hyperlink{examplesreductions}{Examples/Reductions}\tabularnewline
\midrule
\endhead
\protect\hyperlink{building-l2}{Building L2} &
\protect\hyperlink{begin}{Begin} &
\protect\hyperlink{commenting}{Commenting}\tabularnewline
\protect\hyperlink{the-compiler}{The Compiler} &
\protect\hyperlink{literal}{Literal} &
\protect\hyperlink{dereferencing}{Dereferencing}\tabularnewline
\textbf{\protect\hyperlink{syntactic-sugar}{Syntactic Sugar}} &
\protect\hyperlink{reference}{Reference} &
\protect\hyperlink{numbers}{Numbers}\tabularnewline
\textbf{\protect\hyperlink{internal-representation}{Internal
Representation}} & \protect\hyperlink{storage}{Storage} &
\protect\hyperlink{backquoting}{Backquoting}\tabularnewline
& \protect\hyperlink{if}{If} &
\protect\hyperlink{variable-binding}{Variable Binding}\tabularnewline
& \protect\hyperlink{function}{Function} &
\protect\hyperlink{boolean-expressions}{Boolean
Expressions}\tabularnewline
& \protect\hyperlink{invoke}{Invoke} &
\protect\hyperlink{switch-expression}{Switch Expression}\tabularnewline
& \protect\hyperlink{with}{With} &
\protect\hyperlink{characters}{Characters}\tabularnewline
& \protect\hyperlink{continuation}{Continuation} &
\protect\hyperlink{strings}{Strings}\tabularnewline
& \protect\hyperlink{jump}{Jump} &
\protect\hyperlink{closures}{Closures}\tabularnewline
& \protect\hyperlink{meta}{Meta} &
\protect\hyperlink{assume}{Assume}\tabularnewline
& & \protect\hyperlink{fields}{Fields}\tabularnewline
\bottomrule
\end{longtable}

\hypertarget{getting-started}{%
\subsection{Getting Started}\label{getting-started}}

\hypertarget{building-l2}{%
\subsubsection{Building L2}\label{building-l2}}

\begin{verbatim}
./build_bootstrap
./build_selfhost
\end{verbatim}

In this project there are two implementations of L2 compilers. One
implementation is the bootstrap compiler that comprises 3600 lines of C
code which compiles in under a second. The other implementation is a
self-hosting compiler written in about 3600 lines of L2 code (the
meta-program accounts for about 1100 lines and the program accounts for
the other 2500 lines) which compiles in under 9 seconds. Both of them
produce identical object code (modulo padding bytes in the ELFs) when
given identical inputs. \textbf{The bootstrap compiler needs a Linux
distribution running on the x86-64 architecture with the GNU C compiler
installed to be compiled successfully.} To build the bootstrap compiler,
simply run the \texttt{build\_bootstrap} script at the root of the
repository. This will create a directory called \texttt{bin} containing
the file \texttt{l2compile}. \texttt{l2compile} is a compiler of L2 code
and its interface is described in the next section. To build the
self-hosting compiler, simply run the \texttt{build\_selfhost} script at
the root of the repository. This will replace \texttt{l2compile} with a
new compiler that has the same command line interface.

\hypertarget{the-compiler}{%
\subsubsection{The Compiler}\label{the-compiler}}

\begin{verbatim}
./bin/l2compile (metaprogram.o | metaprogram.l2) ... - program.l2 ...
\end{verbatim}

L2 projects are composed of two parts: the program and the metaprogram.
The program is the end product; the stuff that you want in the output
binaries. The metaprogram is the code that the compiler delegates to
during the preprocessing of the program code. The L2 compiler begins by
loading the metaprogram into memory. For the parts of the metaprogram
that are object files, the loading is straightforward. For the parts of
the metaprogram that are L2 files, they cannot simply be compiled and
loaded as they may also need to be preprocessed. Hence a lazy
compilation scheme is implemented where an object file exposing the same
global symbols as the L2 file is loaded, and only later on when one of
its functions is actually called will the compilation of the
corresponding L2 code actually be done. The important gain to doing this
is that the aforementioned compilation now happens in the environment of
the entire metaprogram, that is, the metaprogram can use its entire self
to preprocess itself. Once the metaprogram is loaded, its parts are
linked together and to the compiler's interface for metaprogramming. And
finally each part of the program is compiled into an object file with
the assistance of the metaprogram.

\hypertarget{example}{%
\paragraph{Example}\label{example}}

\hypertarget{file1.l2}{%
\subparagraph{file1.l2}\label{file1.l2}}

\begin{verbatim}
(function foo (frag buf) [@fst[get frag]])

\end{verbatim}

\hypertarget{file2.l2}{%
\subparagraph{file2.l2}\label{file2.l2}}

\begin{verbatim}
(function bar ()
    [putchar (literal 0...01100011)])
(foo [putchar (literal 0...01100110)])
[putchar (literal 0...01100100)]
\end{verbatim}

Running
\texttt{./bin/l2compile\ "./bin/x86\_64.o"\ file1.l2\ -\ file2.l2}
should produce an object file file2.o. file2.o when called should invoke
the function \texttt{putchar} with the ASCII character 'f' and then it
should invoke the function \texttt{putchar} with the ASCII character
'd'. And if its function \texttt{bar} should be called, then it will
call the function \texttt{putchar} with 'c'. Why is it that the first
invocations happen? Because object code resulting from L2 sources are
executed from top to bottom when they are called and because the
expression \texttt{(foo\ {[}putchar\ (literal\ 0...01100110){]})} turned
into \texttt{{[}putchar\ (literal\ 0...01100110){]}}. Why is it that the
aforementioned transformation happened? Because
\texttt{(foo\ {[}putchar\ (literal\ 0...01100110){]})} is a
meta-expression and by the definition of the language causes the
function \texttt{foo} in the metaprogram to be called with the fragment
\texttt{({[}putchar\ (literal\ 0...01100110){]})} as an argument and the
thing which \texttt{foo} then did was to return the first element of
this fragment, \texttt{{[}putchar\ (literal\ 0...01100110){]}}, which
then replaced the original
\texttt{(foo\ {[}putchar\ (literal\ 0...01100110){]})}.

\hypertarget{expressions}{%
\subsection{Expressions}\label{expressions}}

\hypertarget{begin}{%
\subsubsection{Begin}\label{begin}}

\begin{verbatim}
(begin expression1 expression2 ... expressionN)
\end{verbatim}

Evaluates its subexpressions sequentially from left to right. That is,
it evaluates \texttt{expression1}, then \texttt{expression2}, and so on,
ending with the execution of \texttt{expressionN}. Specifying zero
subexpressions is valid. The return value is unspecified.

This expression is implemented by emitting the instructions for
\texttt{expression1}, then emitting the instructions for
\texttt{expression2} immediately afterwords and so on, ending with the
emission of \texttt{expressionN}.

Say the expression \texttt{{[}foo{]}} prints the text "foo" to standard
output and the expression \texttt{{[}bar{]}} prints the text "bar" to
standard output. Then
\texttt{(begin\ {[}foo{]}\ {[}bar{]}\ {[}foo{]}\ {[}foo{]}\ {[}foo{]})}
prints the text "foobarfoofoofoo" to standard output.

\hypertarget{literal}{%
\subsubsection{Literal}\label{literal}}

\begin{verbatim}
(literal b63b62...b0)
\end{verbatim}

The resulting value is the 64 bit number specified in binary inside the
brackets. Specifying less than or more than 64 bits is an error. Useful
for implementing character and string literals, and numbers in other
bases.

This expression is implemented by emitting an instruction to
\texttt{mov} an immediate value into a memory location designated by the
surrounding expression.

Say the expression \texttt{{[}putchar\ x{]}} prints the character
\texttt{x}. Then \texttt{{[}putchar\ (literal\ 0...01100001){]}} prints
the text "a" to standard output.

\hypertarget{reference}{%
\subsubsection{Reference}\label{reference}}

\begin{verbatim}
reference0
\end{verbatim}

The resulting value is the address in memory to which this reference
refers.

This expression is implemented by the emission of an instruction to
\texttt{lea} of some data into a memory location designated by the
surrounding expression.

Say the expression \texttt{{[}get\ x{]}} evaluates to the value at the
reference \texttt{x} and the expression \texttt{{[}set\ x\ y{]}} puts
the value \texttt{y} into the reference \texttt{x}. Then
\texttt{(begin\ {[}set\ x\ (literal\ 0...01100001){]}\ {[}putchar\ {[}get\ x{]}{]})}
prints the text "a" to standard output.

\hypertarget{storage}{%
\subsubsection{Storage}\label{storage}}

\begin{verbatim}
(storage storage0 expression1 expression2 ... expressionN)
\end{verbatim}

If this expression occurs inside a function, then space enough for
\texttt{N} contiguous values has already been reserved in its stack
frame. If it is occuring outside a function, then static memory instead
has been reserved. \texttt{storage0} is a reference to the beginning of
this space. This expression evaluates each of its sub-expressions in an
environment containing \texttt{storage0} and stores the resulting values
in contiguous locations of memory beginning at \texttt{storage0} in the
same order as they were specified. The resulting value of this
expression is \texttt{storage0}.

\texttt{N} contiguous words must be reserved in the current function's
stack-frame plan. The expression is implemented by first emitting the
instructions for any of the subexpressions with the location of the
resulting value fixed to the corresponding reserved word. The same is
done with the remaining expressions repeatedly until the instructions
for all the subexpressions have been emitted. And then second emitting
an instruction to \texttt{lea} of the beginning of the contiguous words
into a memory location designated by the surrounding expression.

The expression
\texttt{{[}putchar\ {[}get\ (storage\ \_\ (literal\ 0...01100001)){]}{]}},
for example, prints the text "a" to standard output.

\hypertarget{if}{%
\subsubsection{If}\label{if}}

\begin{verbatim}
(if expression0 expression1 expression2)
\end{verbatim}

If \texttt{expression0} is non-zero, then only \texttt{expression1} is
evaluated and its resulting value becomes that of the whole expression.
If \texttt{expression0} is zero, then only \texttt{expression2} is
evaluated and its resulting value becomes that of the whole expression.

This expression is implemented by first emitting an instruction to
\texttt{or} \texttt{expression0} with itself. Then an instruction to
\texttt{je} to \texttt{expression2}'s label is emitted. Then the
instructions for \texttt{expression1} are emitted with the location of
the resulting value fixed to the same memory address designated for the
resulting value of the \texttt{if} expression. Then an instruction is
emitted to \texttt{jmp} to the end of all the instructions that are
emitted for this \texttt{if} expression. Then the label for
\texttt{expression2} is emitted. Then the instructions for
\texttt{expression2} are emitted with the location of the resulting
value fixed to the same memory address designated for the resulting
value of the \texttt{if} expression.

The expression
\texttt{{[}putchar\ (if\ (literal\ 0...0)\ (literal\ 0...01100001)\ (literal\ 0...01100010)){]}}
prints the text "b" to standard output.

\hypertarget{function}{%
\subsubsection{Function}\label{function}}

\begin{verbatim}
(function function0 (reference1 reference2 ... referenceN) expression0)
\end{verbatim}

Makes a function to be invoked with exactly \texttt{N} arguments. When
the function is invoked, \texttt{expression0} is evaluated in an
environment where \texttt{function0} is a reference to the function
itself and \texttt{reference1}, \texttt{reference2}, up to
\texttt{referenceN} are references to the resulting values of evaluating
the corresponding arguments in the invoke expression invoking this
function. Once the evaluation is complete, control flow returns to the
invoke expression and the invoke expression's resulting value is the
resulting value of evaluating \texttt{expression0}. The resulting value
of this function expression is a reference to the function.

This expression is implemented by first emitting an instruction to
\texttt{mov} the address \texttt{function0} (a label to be emitted
later) into the memory location designated by the surrounding
expression. Then an instruction is emitted to \texttt{jmp} to the end of
all the instructions that are emitted for this function. Then the label
named \texttt{function0} is emitted. Then instructions to \texttt{push}
each callee-saved register onto the stack are emitted. Then an
instruction to push the frame-pointer onto the stack is emitted. Then an
instruction to move the value of the stack-pointer into the
frame-pointer is emitted. Then an instruction to \texttt{sub} from the
stack-pointer the amount of words reserved on this function's
stack-frame is emitted. After this the instructions for
\texttt{expression0} are emitted with the location of the resulting
value fixed to a word within the stack-pointer's drop. After this an
instruction is emitted to \texttt{mov} the word from this location into
the register \texttt{eax}. And finally, instructions are emitted to
\texttt{leave} the current function's stack-frame, \texttt{pop} the
callee-save registers, and \texttt{ret} to the address of the caller.

The expression
\texttt{{[}putchar\ {[}(function\ my-\ (a\ b)\ {[}-\ {[}get\ b{]}\ {[}get\ a{]}{]})\ (literal\ 0...01)\ (literal\ 0...01100011){]}{]}}
prints the text "b" to standard output.

\hypertarget{invoke}{%
\subsubsection{Invoke}\label{invoke}}

\begin{verbatim}
(invoke function0 expression1 expression2 ... expressionN)
[function0 expression1 expression2 ... expressionN]
\end{verbatim}

Both the above expressions are equivalent. Evaluates \texttt{function0},
\texttt{expression1}, \texttt{expression2}, up to \texttt{expressionN}
in an unspecified order and then invokes \texttt{function0}, a reference
to a function, providing it with the resulting values of evaluating
\texttt{expression1} up to \texttt{expressionN}, in order. The resulting
value of this expression is determined by the function being invoked.

\texttt{N+1} words must be reserved in the current function's
stack-frame plan. The expression is implemented by emitting the
instructions for any of the subexpressions with the location of the
resulting value fixed to the corresponding reserved word. The same is
done with the remaining expressions repeatedly until the instructions
for all the subexpressions have been emitted. Then an instruction to
\texttt{push} the last reserved word onto the stack is emitted, followed
by the second last, and so on, ending with an instruction to
\texttt{push} the first reserved word onto the stack. A \texttt{call}
instruction with the zeroth reserved word as the operand is then
emitted. Note that L2 expects registers \texttt{esp}, \texttt{ebp},
\texttt{ebx}, \texttt{esi}, and \texttt{edi} to be preserved across
\texttt{call}s. An \texttt{add} instruction that pops N words off the
stack is then emitted. Then an instruction is emitted to \texttt{mov}
the register \texttt{eax} into a memory location designated by the
surrounding expression.

A function with the reference \texttt{-} that returns the value of
subtracting its second parameter from its first could be defined as
follows:

\begin{verbatim}
-:
movl 4(%esp), %eax
subl 8(%esp), %eax
ret
\end{verbatim}

The following invocation of it,
\texttt{(invoke\ putchar\ (invoke\ -\ (literal\ 0...01100011)\ (literal\ 0...01)))},
prints the text "b" to standard output.

\hypertarget{with}{%
\subsubsection{With}\label{with}}

\begin{verbatim}
(with continuation0 expression0)
\end{verbatim}

Makes a continuation to the containing expression that is to be
\texttt{jump}ed to with exactly one argument. Then \texttt{expression0}
is evaluated in an environment where \texttt{continuation0} is a
reference to the aforementioned continuation. The resulting value of
this expression is unspecified if the evaluation of \texttt{expression0}
completes. If the continuation \texttt{continuation0} is \texttt{jump}ed
to, then this \texttt{with} expression evaluates to the resulting value
of the single argument within the responsible \texttt{jump} expression.

5+1 words must be reserved in the current function's stack-frame plan.
Call the reference to the first word of the reservation
\texttt{continuation0}. This expression is implemented by first emitting
instructions to store the program's state at \texttt{continuation0},
that is, instructions are emitted to \texttt{mov} \texttt{ebp}, the
address of the instruction that should be executed after continuing (a
label to be emitted later), \texttt{edi}, \texttt{esi}, and
\texttt{ebx}, in that order, to the first 5 words at
\texttt{continuation0}. After this, the instructions for
\texttt{expression0} are emitted. Then the label for the first
instruction of the continuation is emitted. And finally, an instruction
is emitted to \texttt{mov} the resulting value of the continuation, the
6th word at \texttt{continuation0}, into the memory location designated
by the surrounding expression.

\hypertarget{examples}{%
\paragraph{Examples}\label{examples}}

Note that the expression \texttt{\{continuation0\ expression0\}}
\texttt{jump}s to the continuation reference given by
\texttt{continuation0} with resulting value of evaluating
\texttt{expression0} as its argument. With the note in mind, the
expression
\texttt{(begin\ {[}putchar\ (with\ ignore\ (begin\ \{ignore\ (literal\ 0...01001110)\}\ {[}foo{]}\ {[}foo{]}\ {[}foo{]})){]}\ {[}bar{]})}
prints the text "nbar" to standard output.

The following assembly function \texttt{allocate} receives the number of
bytes it is to allocate as its first argument, allocates that memory,
and passes the initial address of this memory as the single argument to
the continuation it receives as its second argument.

\begin{verbatim}
allocate:
/* All sanctioned by L2 ABI: */
movl 8(%esp), %ecx
movl 16(%ecx), %ebx
movl 12(%ecx), %esi
movl 8(%ecx), %edi
movl 0(%ecx), %ebp
subl 4(%esp), %esp
andl $0xFFFFFFFC, %esp
movl %esp, 20(%ecx)
jmp *4(%ecx)
\end{verbatim}

The following usage of it,
\texttt{(with\ dest\ {[}allocate\ (literal\ 0...011)\ dest{]})},
evaluates to the address of the allocated memory. If allocate had just
decreased \texttt{esp} and returned, it would have been invalid because
L2 expects functions to preserve \texttt{esp}.

\hypertarget{continuation}{%
\subsubsection{Continuation}\label{continuation}}

\begin{verbatim}
(continuation continuation0 (reference1 reference2 ... referenceN) expression0)
\end{verbatim}

Makes a continuation to be \texttt{jump}ed to with exactly \texttt{N}
arguments. When the continuation is \texttt{jump}ed to,
\texttt{expression0} is evaluated in an environment where
\texttt{continuation0} is a reference to the continuation itself and
\texttt{reference1}, \texttt{reference2}, up to \texttt{referenceN} are
references to the resulting values of evaluating the corresponding
arguments in the \texttt{jump} expression \texttt{jump}ing to this
function. Undefined behavior occurs if the evaluation of
\texttt{expression0} completes - i.e. programmer must direct the control
flow out of \texttt{continuation0} somewhere within
\texttt{expression0}. The resulting value of this \texttt{continuation}
expression is a reference to the continuation.

5+N words must be reserved in the current function's stack-frame plan.
Call the reference to the first word of the reservation
\texttt{continuation0}. This expression is implemented by first emitting
an instruction to \texttt{mov} the reference \texttt{continuation0} into
the memory location designated by the surrounding expression.
Instructions are then emitted to store the program's state at
\texttt{continuation0}, that is, instructions are emitted to
\texttt{mov} \texttt{ebp}, the address of the instruction that should be
executed after continuing (a label to be emitted later), \texttt{edi},
\texttt{esi}, and \texttt{ebx}, in that order, to the first 5 words at
\texttt{continuation0}. Then an instruction is emitted to \texttt{jmp}
to the end of all the instructions that are emitted for this
\texttt{continuation} expression. Then the label for the first
instruction of the continuation is emitted. After this the instructions
for \texttt{expression0} are emitted.

The expression
\texttt{\{(continuation\ forever\ (a\ b)\ (begin\ {[}putchar\ {[}get\ a{]}{]}\ {[}putchar\ {[}get\ b{]}{]}\ \{forever\ {[}-\ {[}get\ a{]}\ (literal\ 0...01){]}\ {[}-\ {[}get\ b{]}\ (literal\ 0...01){]}\}))\ (literal\ 0...01011010)\ (literal\ 0...01111010)\}}
prints the text "ZzYyXxWw"... to standard output.

\hypertarget{jump}{%
\subsubsection{Jump}\label{jump}}

\begin{verbatim}
(jump continuation0 expression1 expression2 ... expressionN)
{continuation0 expression1 expression2 ... expressionN}
\end{verbatim}

Both the above expressions are equivalent. Evaluates
\texttt{continuation0}, \texttt{expression1}, \texttt{expression2}, up
to \texttt{expressionN} in an unspecified order and then \texttt{jump}s
to \texttt{continuation0}, a reference to a continuation, providing it
with a local copies of \texttt{expression1} up to \texttt{expressionN}
in order. The resulting value of this expression is unspecified.

\texttt{N+1} words must be reserved in the current function's
stack-frame plan. The expression is implemented by emitting the
instructions for any of the subexpressions with the location of the
resulting value fixed to the corresponding reserved word. The same is
done with the remaining expressions repeatedly until the instructions
for all the subexpressions have been emitted. Then an instruction to
\texttt{mov} the first reserved word to 5 words from the beginning of
the continuation is emitted, followed by an instruction to \texttt{mov}
the second reserved word to an address immediately after that, and so
on, ending with an instruction to \texttt{mov} the last reserved word
into the last memory address of that area. The program's state, that is,
\texttt{ebp}, the address of the instruction that should be executed
after continuing, \texttt{edi}, \texttt{esi}, and \texttt{ebx}, in that
order, are what is stored at the beginning of a continuation.
Instructions to \texttt{mov} these values from the buffer into the
appropriate registers and then set the program counter appropriately
are, at last, emitted.

The expression
\texttt{(begin\ (with\ cutter\ (jump\ (continuation\ cuttee\ ()\ (begin\ {[}bar{]}\ {[}bar{]}\ (jump\ cutter\ (literal\ 0...0))\ {[}bar{]}\ {[}bar{]}\ {[}bar{]}))))\ {[}foo{]})}
prints the text "barbarfoo" to standard output.

\hypertarget{an-optimization}{%
\paragraph{An Optimization}\label{an-optimization}}

Looking at the examples above where the continuation reference does not
escape, \texttt{(with\ reference0\ expression0)} behaves a lot like the
pseudo-assembly \texttt{expression0\ reference0:} and
\texttt{(continuation\ reference0\ (...)\ expression0)} behaves a lot
like \texttt{reference0:\ expression0}. To be more precise, when
references to a particular continuation only occur as the
\texttt{continuation0} subexpression of a \texttt{jump} statement, we
know that the continuation is constrained to the function in which it is
declared, and hence there is no need to store or restore \texttt{ebp},
\texttt{edi}, \texttt{esi}, and \texttt{ebx}. Continuations, then, are
how efficient iteration is achieved in L2.

\hypertarget{syntactic-sugar}{%
\subsection{Syntactic Sugar}\label{syntactic-sugar}}

\hypertarget{ux24a1...an}{%
\subsubsection{\texorpdfstring{\texttt{\$a1...aN}}{\$a1...aN}}\label{ux24a1...an}}

In what follows, it is assumed that \texttt{\$a1...aN} is not part of a
larger string. If \texttt{\$a1...aN} is simply a \texttt{\$}, then it
remains unchanged. Otherwise at least a character follows the
\texttt{\$}; in this case \texttt{\$a1...aN} turns into
\texttt{(\$\ a1...aN)}.

For example, the expression \texttt{\$\$hello\$bye} turns into
\texttt{(\$\ \$hello\$bye)} which turns into
\texttt{(\$\ (\$\ hello\$bye))}

\hypertarget{ux5cux23a1...anux2c-ux2ca1...anux2c-ux60a1...an}{%
\subsubsection{\texorpdfstring{\texttt{\#a1...aN}, \texttt{,a1...aN},
\texttt{\textasciigrave{}a1...aN}}{\#a1...aN, ,a1...aN, `a1...aN}}\label{ux5cux23a1...anux2c-ux2ca1...anux2c-ux60a1...an}}

Analogous transformations to the one for \texttt{\$a1...aN} happen.

\hypertarget{internal-representation}{%
\subsection{Internal Representation}\label{internal-representation}}

After substituting out the syntactic sugar defined in the
\protect\hyperlink{invoke}{invoke}, \protect\hyperlink{jump}{jump}, and
\protect\hyperlink{syntactic-sugar}{syntactic sugar} sections, we find
that all L2 programs are just fragments where a fragment is either a
symbol or a list of fragments. And furthermore, every symbol can be seen
as a list of its characters so that for example \texttt{foo} becomes
\texttt{(f\ o\ o)}. The following functions that manipulate these
fragments are not part of the L2 language and hence the compiler does
not give references to them special treatment during compilation.
However, when they are used in an L2 meta-program, undefined references
to these functions are to be resolved by the compiler.

\hypertarget{ux7bux5bux7dlst-x-y-bux7bux5dux7d}{%
\subsubsection{\texorpdfstring{\texttt{{[}lst\ x\ y\ b{]}}}{{[}lst x y b{]}}}\label{ux7bux5bux7dlst-x-y-bux7bux5dux7d}}

\texttt{y} must be a list and \texttt{b} a buffer.

Makes a list where \texttt{x} is first and \texttt{y} is the rest in the
buffer \texttt{b}.

Say the fragment \texttt{foo} is stored at \texttt{a} and the list
\texttt{(bar)} is stored at \texttt{b}. Then
\texttt{{[}lst\ {[}get\ a{]}\ {[}get\ b{]}{]}} is the fragment
\texttt{(foo\ bar)}.

\hypertarget{ux7bux5bux7dsymbolux3f-xux7bux5dux7d}{%
\subsubsection{\texorpdfstring{\texttt{{[}symbol?\ x{]}}}{{[}symbol? x{]}}}\label{ux7bux5bux7dsymbolux3f-xux7bux5dux7d}}

\texttt{x} must be a fragment.

Evaluates to the one if \texttt{x} is also a symbol. Otherwise evaluates
to zero.

Say the fragment \texttt{foo} is stored at \texttt{a}. Then
\texttt{{[}symbol?\ {[}get\ a{]}{]}} evaluates to
\texttt{(literal\ 0...01)}.

\hypertarget{ux7bux5bux7dux40fst-xux7bux5dux7d}{%
\subsubsection{\texorpdfstring{\texttt{{[}@fst\ x{]}}}{{[}@fst x{]}}}\label{ux7bux5bux7dux40fst-xux7bux5dux7d}}

\texttt{x} must be a list.

Evaluates to the first of \texttt{x}.

Say the list \texttt{foo} is stored at \texttt{a}. Then
\texttt{{[}@fst\ {[}get\ a{]}{]}} is the character \texttt{f}. This
\texttt{f} is not a list but is a character.

\hypertarget{ux7bux5bux7dux40rst-xux7bux5dux7d}{%
\subsubsection{\texorpdfstring{\texttt{{[}@rst\ x{]}}}{{[}@rst x{]}}}\label{ux7bux5bux7dux40rst-xux7bux5dux7d}}

\texttt{x} must be a list.

Evaluates to a list that is the rest of \texttt{x}.

Say the list \texttt{foo} is stored at \texttt{a}. Then
\texttt{{[}@rst\ {[}get\ a{]}{]}} is the fragment \texttt{oo}.

\hypertarget{emt}{%
\subsubsection{\texorpdfstring{\texttt{emt}}{emt}}\label{emt}}

Evaluates to the empty list.

Say the fragment \texttt{foo} is stored at \texttt{a}. Then
\texttt{{[}lst\ {[}get\ a{]}\ emt{]}} is the fragment \texttt{(foo)}.

\hypertarget{ux7bux5bux7demtux3f-xux7bux5dux7d}{%
\subsubsection{\texorpdfstring{\texttt{{[}emt?\ x{]}}}{{[}emt? x{]}}}\label{ux7bux5bux7demtux3f-xux7bux5dux7d}}

\texttt{x} must be a list.

Evaluates to the one if \texttt{x} is the empty list. Otherwise
evaluates to zero.

\texttt{{[}emt?\ emt{]}} evaluates to \texttt{(literal\ 0...01)}.

\hypertarget{-ux5ctextlessux7bux7dcharacterux5ctextgreaterux7bux7d-}{%
\subsubsection{\texorpdfstring{\texttt{-\textless{}character\textgreater{}-}}{-\textless{}character\textgreater{}-}}\label{-ux5ctextlessux7bux7dcharacterux5ctextgreaterux7bux7d-}}

Evaluates to the character \texttt{\textless{}character\textgreater{}}.

Say that a buffer is stored at \texttt{b}. Then the expression
\texttt{{[}lst\ -f-\ {[}lst\ -o-\ {[}lst\ -o-\ emt\ {[}get\ b{]}{]}\ {[}get\ b{]}{]}\ {[}get\ b{]}{]}}
evaluates to the fragment \texttt{foo}.

\hypertarget{ux7bux5bux7dchar=-x-yux7bux5dux7d}{%
\subsubsection{\texorpdfstring{\texttt{{[}char=\ x\ y{]}}}{{[}char= x y{]}}}\label{ux7bux5bux7dchar=-x-yux7bux5dux7d}}

\texttt{x} and \texttt{y} must be characters.

Evaluates to one if \texttt{x} is the same character as \texttt{y},
otherwise it evaluates to zero.

Say the character \texttt{d} is stored at both \texttt{x} and
\texttt{y}. Then \texttt{{[}char=\ {[}get\ x{]}\ {[}get\ y{]}{]}}
evaluates to \texttt{(literal\ 0...01)}.

\hypertarget{ux7bux5bux7dbegin-x-bux7bux5dux7d}{%
\subsubsection{\texorpdfstring{\texttt{{[}begin\ x\ b{]}}}{{[}begin x b{]}}}\label{ux7bux5bux7dbegin-x-bux7bux5dux7d}}

\texttt{x} must be a list of fragments and \texttt{b} a buffer.

Evaluates to an fragment formed by prepending the symbol \texttt{begin}
to \texttt{x}. The \texttt{begin} function could have the following
definition:
\texttt{(function\ begin\ (frags\ b)\ {[}lst\ {[}lst\ -b-\ {[}lst\ -e-\ {[}lst\ -g-\ {[}lst\ -i-\ {[}lst\ -n-\ emt\ {[}get\ b{]}{]}\ {[}get\ b{]}{]}\ {[}get\ b{]}{]}\ {[}get\ b{]}{]}\ {[}get\ b{]}{]}\ {[}get\ frags{]}\ {[}get\ b{]}{]})}.

\hypertarget{ux7bux5bux7dliteral-x-bux7bux5dux7dux2c-ux7bux5bux7dstorage-x-bux7bux5dux7dux2c-ux7bux5bux7dif-x-bux7bux5dux7dux2c-ux7bux5bux7dfunction-x-bux7bux5dux7dux2c-ux7bux5bux7dinvoke-x-bux7bux5dux7dux2c-ux7bux5bux7dwith-x-bux7bux5dux7dux2c-ux7bux5bux7dcontinuation-x-bux7bux5dux7dux2c-ux7bux5bux7djump-x-bux7bux5dux7d}{%
\subsubsection{\texorpdfstring{\texttt{{[}literal\ x\ b{]}},
\texttt{{[}storage\ x\ b{]}}, \texttt{{[}if\ x\ b{]}},
\texttt{{[}function\ x\ b{]}}, \texttt{{[}invoke\ x\ b{]}},
\texttt{{[}with\ x\ b{]}}, \texttt{{[}continuation\ x\ b{]}},
\texttt{{[}jump\ x\ b{]}}}{{[}literal x b{]}, {[}storage x b{]}, {[}if x b{]}, {[}function x b{]}, {[}invoke x b{]}, {[}with x b{]}, {[}continuation x b{]}, {[}jump x b{]}}}\label{ux7bux5bux7dliteral-x-bux7bux5dux7dux2c-ux7bux5bux7dstorage-x-bux7bux5dux7dux2c-ux7bux5bux7dif-x-bux7bux5dux7dux2c-ux7bux5bux7dfunction-x-bux7bux5dux7dux2c-ux7bux5bux7dinvoke-x-bux7bux5dux7dux2c-ux7bux5bux7dwith-x-bux7bux5dux7dux2c-ux7bux5bux7dcontinuation-x-bux7bux5dux7dux2c-ux7bux5bux7djump-x-bux7bux5dux7d}}

These functions are analogous to \texttt{begin}.

\hypertarget{expressions-continued}{%
\subsection{Expressions Continued}\label{expressions-continued}}

\hypertarget{meta}{%
\subsubsection{Meta}\label{meta}}

\begin{verbatim}
(function0 expression1 ... expressionN)
\end{verbatim}

If the above expression is not listed above, then \texttt{function0}
from the metaprogram is invoked with the (unevaluated) list of
\protect\hyperlink{internal-representation}{fragments}
\texttt{(expression1\ expression2\ ...\ expressionN)} as its first
argument and a buffer in which the replacement is to be constructed as
its second argument. The fragment returned by this function then
replaces the entire fragment
\texttt{(function0\ expression1\ ...\ expressionN)}. If the result of
this replacement contains a meta-expression, then the above process is
repeated. When this process terminates, the appropriate assembly code
for the resulting expression is emitted.

Meta-expressions were already demonstrated in the
\protect\hyperlink{the-compiler}{compiler section}.

\hypertarget{examplesux2freductions}{%
\subsection{Examples/Reductions}\label{examplesux2freductions}}

In the extensive list processing that follows in this section, the
following functions prove to be convenient abbreviations:

\hypertarget{abbreviations.l2}{%
\paragraph{abbreviations.l2}\label{abbreviations.l2}}

\begin{verbatim}
(function @frst (l) [@fst [@rst [get l]]])
(function @ffrst (l) [@fst [@frst [get l]]])
(function @frfrst (l) [@fst [@rst [@frst [get l]]]])
(function @rrst (l) [@rst [@rst [get l]]])
(function @rrrst (l) [@rst [@rrst [get l]]])
(function @rfst (l) [@rst [@fst [get l]]])
(function @frfst (l) [@fst [@rfst [get l]]])
(function @frrfst (l) [@fst [@rst [@rfst [get l]]]])
(function @frrst (l) [@fst [@rst [@rst [get l]]]])
(function @frrrst (l) [@fst [@rst [@rst [@rst [get l]]]]])
(function @frrrrst (l) [@fst [@rst [@rst [@rst [@rst [get l]]]]]])
(function @frrrrrst (l) [@fst [@rst [@rst [@rst [@rst [@rst [get l]]]]]]])
(function @ffst (l) [@fst [@fst [get l]]])
(function llst (a b c r) [lst [get a] [lst [get b] [get c] [get r]] [get r]])
(function lllst (a b c d r) [lst [get a] [llst [get b] [get c] [get d] [get r]] [get r]])
(function llllst (a b c d e r) [lst [get a] [lllst [get b] [get c] [get d] [get e] [get r]]
    [get r]])
(function lllllst (a b c d e f r) [lst [get a] [llllst [get b] [get c] [get d] [get e]
    [get f] [get r]] [get r]])
(function llllllst (a b c d e f g r) [lst [get a] [lllllst [get b] [get c] [get d] [get e]
    [get f] [get g] [get r]] [get r]])
(function lllllllst (a b c d e f g h r) [lst [get a] [llllllst [get b] [get c]
    [get d] [get e] [get f] [get g] [get h] [get r]] [get r]])
\end{verbatim}

\hypertarget{commenting}{%
\subsubsection{Commenting}\label{commenting}}

L2 has no built-in mechanism for commenting code written in it. The
following comment function takes a list of fragments as its argument and
returns an empty begin expression effectively causing its arguments to
be ignored. Its implementation and use follows:

\hypertarget{comments.l2}{%
\paragraph{comments.l2}\label{comments.l2}}

\begin{verbatim}
(function ;; (l r) [lst [lllllst -b- -e- -g- -i- -n- emt $r] emt $r])
\end{verbatim}

\hypertarget{test1.l2}{%
\paragraph{test1.l2}\label{test1.l2}}

\begin{verbatim}
(;; This is a comment, take no notice.)
\end{verbatim}

\hypertarget{shell}{%
\paragraph{shell}\label{shell}}

\begin{verbatim}
./bin/l2compile "bin/x86_64.o" abbreviations.l2 comments.l2 - test1.l2
\end{verbatim}

\hypertarget{dereferencing}{%
\subsubsection{Dereferencing}\label{dereferencing}}

So far, we have been writing \texttt{{[}get\ x{]}} in order to get the
value at the address \texttt{x}. Given the frequency with which
dereferencing happens, writing \texttt{{[}get\ x{]}} all the time can
greatly increase the amount of code required to get a task done. The
following function \texttt{\$} implements an abbreviation for
dereferencing.

\hypertarget{dereference.l2}{%
\paragraph{dereference.l2}\label{dereference.l2}}

\begin{verbatim}
(function $ (var r)
    [llst
        [llllllst -i- -n- -v- -o- -k- -e- emt [get r]]
        [lllst -g- -e- -t- emt [get r]]
        [get var][get r]])
\end{verbatim}

\hypertarget{test2.l2}{%
\paragraph{test2.l2}\label{test2.l2}}

\begin{verbatim}
(storage a (literal 0...01000001))
(storage c a)
[putchar $$c]
\end{verbatim}

\hypertarget{or-equivalently}{%
\subparagraph{or equivalently}\label{or-equivalently}}

\begin{verbatim}
(storage a (literal 0...01000001))
(storage c a)
[putchar [get [get c]]]
\end{verbatim}

\hypertarget{shell-1}{%
\paragraph{shell}\label{shell-1}}

\begin{verbatim}
./bin/l2compile "bin/x86_64.o" abbreviations.l2 comments.l2 dereference.l2 - test2.l2
\end{verbatim}

Note that in the above code that \texttt{a} and \texttt{c} have global
scope. This is because the storage expressions are top-level.

\hypertarget{numbers}{%
\subsubsection{Numbers}\label{numbers}}

Integer literals prove to be quite tedious in L2 as can be seen from
some of the examples in the expressions section. The following function,
\texttt{\#}, implements decimal arithmetic for x86-64 by reading in a
symbol in base 10 and writing out the equivalent fragment in base 2:

\hypertarget{numbers64.l2}{%
\paragraph{numbers64.l2}\label{numbers64.l2}}

\begin{verbatim}
(;; Turns an 8-byte value into a literal-expression representation of it.)

(function value->literal (binary r)
    [lst [lllllllst -l- -i- -t- -e- -r- -a- -l- emt $r]
        [lst (with return {(continuation write (count in out)
                (if $count
                    {write [- $count (literal 0...01)]
                        [>> $in (literal 0...01)]
                        [lst (if [land $in (literal 0...01)]
                            -1- -0-) $out $r]}
                    {return $out}))
                (literal 0...01000000) $binary emt})
            emt $r]$r])

(;; Turns the base-10 fragment input into a literal expression.)

(function # (l r) [value->literal
    (with return {(continuation read (in out)
        (if [emt? $in]
            {return $out}
            {read [@rst $in] [+ [* $out (literal 0...01010)]
                (if [char= [@fst $in] -9-] (literal 0...01001)
                (if [char= [@fst $in] -8-] (literal 0...01000)
                (if [char= [@fst $in] -7-] (literal 0...0111)
                (if [char= [@fst $in] -6-] (literal 0...0110)
                (if [char= [@fst $in] -5-] (literal 0...0101)
                (if [char= [@fst $in] -4-] (literal 0...0100)
                (if [char= [@fst $in] -3-] (literal 0...011)
                (if [char= [@fst $in] -2-] (literal 0...010)
                (if [char= [@fst $in] -1-] (literal 0...01)
                    (literal 0...0))))))))))]}))
        [@fst $l] (literal 0...0)}) $r])
\end{verbatim}

\hypertarget{test3.l2}{%
\paragraph{test3.l2}\label{test3.l2}}

\begin{verbatim}
[putchar (# 65)]
\end{verbatim}

\hypertarget{or-equivalently-1}{%
\subparagraph{or equivalently}\label{or-equivalently-1}}

\begin{verbatim}
[putchar #65]
\end{verbatim}

\hypertarget{shell-2}{%
\paragraph{shell}\label{shell-2}}

\begin{verbatim}
./bin/l2compile "bin/x86_64.o" abbreviations.l2 comments.l2 dereference.l2 numbers64.l2 - \
    test3.l2
\end{verbatim}

\hypertarget{backquoting}{%
\subsubsection{Backquoting}\label{backquoting}}

The \texttt{foo} example in the internal representation section shows
how tedious writing a function that outputs a symbol can be. The
backquote function reduces this tedium. It takes a fragment and a buffer
as its argument and, generally, it returns a fragment that makes that
fragment. The exception to this rule is that if a sub-expression of its
input fragment is of the form \texttt{(,\ expr0)}, then the fragment
\texttt{expr0} is inserted verbatim into that position of the output
fragment. Backquote can be implemented and used as follows:

\hypertarget{backquote.l2}{%
\paragraph{backquote.l2}\label{backquote.l2}}

\begin{verbatim}
(function ` (l r)
    [(function aux (s t r)
        (if [emt? $s] [lllst -e- -m- -t- emt $r]

        (if (if [emt? $s] #0 (if [symbol? $s] #0 (if [emt? [@fst $s]]
            #0 (if [char= [@ffst $s] -,-] [emt? [@rfst $s]] #0))))
                    [@frst $s]

        [lllllst [llllllst -i- -n- -v- -o- -k- -e- emt $r]
            [lllst -l- -s- -t- emt $r]
                (if [symbol? $s]
                        [lllst --- [@fst $s] --- emt $r]
                        [aux [@fst $s] $t $r])
                    [aux [@rst $s] $t $r] $t emt $r]))) [@fst $l] [@frst $l] $r])
\end{verbatim}

\hypertarget{anotherfunction.l2:}{%
\paragraph{anotherfunction.l2:}\label{anotherfunction.l2:}}

\begin{verbatim}
(function make-A-function (l r)
    (` (function A (,emt) [putchar #65]) $r))
\end{verbatim}

\hypertarget{or-equivalently-2}{%
\subparagraph{or equivalently}\label{or-equivalently-2}}

\begin{verbatim}
(function make-A-function (l)
    (`(function A () [putchar #65])$r))
\end{verbatim}

\hypertarget{test4.l2}{%
\paragraph{test4.l2}\label{test4.l2}}

\begin{verbatim}
[(make-A-function)]
\end{verbatim}

\hypertarget{shell-3}{%
\paragraph{shell}\label{shell-3}}

\begin{verbatim}
./bin/l2compile "bin/x86_64.o" abbreviations.l2 comments.l2 dereference.l2 numbers64.l2 \
    backquote.l2 anotherfunction.l2 - test4.l2
\end{verbatim}

\hypertarget{variable-binding}{%
\subsubsection{Variable Binding}\label{variable-binding}}

Variable binding is enabled by the \texttt{continuation} expression.
\texttt{continuation} is special because, like \texttt{function}, it
allows references to be bound. Unlike \texttt{function}, however,
expressions within \texttt{continuation} can directly access its parent
function's variables. The \texttt{let} binding function implements the
following transformation:

\begin{verbatim}
(let (params args) ... expr0)
->
(with return
    {(continuation templet0 (params ...)
        {return expr0}) vals ...})
\end{verbatim}

It is implemented and used as follows:

\hypertarget{let.l2}{%
\paragraph{let.l2}\label{let.l2}}

\begin{verbatim}
(;; Reverses the given list. $l is the list to be reversed. $r is the buffer into
    which the reversed list will be put. Return value is the reversed list.)

(function meta:reverse (l r)
    (with return
        {(continuation _ (l reversed)
            (if [emt? $l]
                {return $reversed}
                {_ [@rst $l] [lst [@fst $l] $reversed $r]})) $l emt}))

(;; Maps the given list using the given function. $l is the list to be mapped. $ctx
    is always passed as a second argument to the mapper. $mapper is the two argument
    function that will be supplied a list item as its first argument and $ctx as its
    second argument and will return an argument that will be put into the corresponding
    position of another list. $r is the buffer into which the list being constructed
    will be put. Return value is the mapped list.)

(function meta:map (l ctx mapper r)
    (with return
        {(continuation aux (in out)
            (if [emt? $in]
                {return [meta:reverse $out $r]}
                {aux [@rst $in] [lst [$mapper [@fst $in] $ctx] $out $r]})) $l emt}))

(function let (l r)
    (`(with let:return
        (,[llst (` jump $r) (`(continuation let:aux
            (,[meta:map [@rst [meta:reverse $l $r]] (null) @fst $r])
            {let:return (,[@fst [meta:reverse $l $r]])}) $r) [meta:map [@rst
                [meta:reverse $l $r]] (null) @frst $r] $r])) $r))
\end{verbatim}

\hypertarget{test5.l2}{%
\paragraph{test5.l2}\label{test5.l2}}

\begin{verbatim}
(let (x #12) (begin
    (function what? () [printf (" x is %i) $x])
    [what?]
    [what?]
    [what?]))
\end{verbatim}

Note in the above code that \texttt{what?} is only able to access
\texttt{x} because \texttt{x} is defined outside of all functions and
hence is statically allocated. Also note that L2 permits reference
shadowing, so \texttt{let} expressions can be nested without worrying,
for instance, about the impact of an inner \texttt{templet0} on an outer
one.

\hypertarget{shell-4}{%
\paragraph{shell}\label{shell-4}}

\begin{verbatim}
./bin/l2compile "bin/x86_64.o" abbreviations.l2 comments.l2 dereference.l2 numbers64.l2 \
    backquote.l2 let.l2 - test5.l2
\end{verbatim}

\hypertarget{boolean-expressions}{%
\subsubsection{Boolean Expressions}\label{boolean-expressions}}

The Boolean literals true and false are achieved using macros that
return the same literal fragment regardless of the arguments supplied to
them. Short-circut Boolean expressions are achieved through the
\texttt{if} expression. The \texttt{if} expression is special because it
has the property that only two out of its three sub-expressions are
evaluated when it itself is evaluated. Now, the Boolean expressions
implement the following transformations:

\begin{verbatim}
(false) -> (literal #0)

(true) -> (literal #1)

(or expr1 expr2 ... exprN)
->
(let (or:temp expr1) (if $or:temp
    $or:temp
    (let (or:temp expr2) (if $or:temp
        $or:temp
        ...
            (let (or:temp exprN) (if $or:temp
                $or:temp
                (false)))))))

(and expr1 expr2 ... exprN)
->
(let (and:temp expr1) (if $and:temp
    (let (and:temp expr2) (if $and:temp
        ...
            (let (and:temp exprN) (if $and:temp
                (true)
                $and:temp))
        $and:temp))
    $and:temp))

(not expr1)
->
(if expr1 (false) (true))
\end{verbatim}

These transformations are implemented and used as follows:

\hypertarget{boolean.l2}{%
\paragraph{boolean.l2}\label{boolean.l2}}

\begin{verbatim}
(function mk# (r value) [value->literal $value $r])

(function false (l r) [mk# $r #0])

(function true (l r) [mk# $r #1])

(function or (l r) (with return
    {(continuation loop (l sexpr)
            (if [emt? $l]
                {return $sexpr}
                {loop [@rst $l] (`(let (or:temp (,[@fst $l])) (if $or:temp $or:temp
                    (, $sexpr $r)))$r)}))
        [meta:reverse $l $r] (`(false)$r)}))

(function and (l r) (with return
    {(continuation loop (l sexpr)
            (if [emt? $l]
                {return $sexpr}
                {loop [@rst $l] (`(let (and:temp (,[@fst $l])) (if $and:temp
                    (, $sexpr $r) $and:temp))$r)}))
        [meta:reverse $l $r] (`(true)$r)}))

(function not (l r) (`(if (,[@fst $l]) (false) (true))$r))
\end{verbatim}

\hypertarget{test6.l2}{%
\paragraph{test6.l2}\label{test6.l2}}

\begin{verbatim}
(and (false) [/ #1 #0])
\end{verbatim}

\hypertarget{shell-5}{%
\paragraph{shell}\label{shell-5}}

\begin{verbatim}
./bin/l2compile "bin/x86_64.o" abbreviations.l2 comments.l2 dereference.l2 numbers64.l2 \
    backquote.l2 let.l2 boolean.l2 - test6.l2
\end{verbatim}

\hypertarget{switch-expression}{%
\subsubsection{Switch Expression}\label{switch-expression}}

Now we will implement a variant of the switch statement that is
parameterized by an equality predicate. The \texttt{switch} selection
function will, for example, do the following transformation:

\begin{verbatim}
(switch eq0 val0 (vals exprs) ... expr0)
->
(let (tempeq0 eq0) (tempval0 val0)
    (if [[' tempeq0] [' tempval0] vals1]
        exprs1
        (if [[' tempeq0] [' tempval0] vals2]
            exprs2
            ...
                (if [[' tempeq0] [' tempval0] valsN]
                    exprsN
                    expr0))))
\end{verbatim}

It is implemented and used as follows:

\hypertarget{switch.l2}{%
\paragraph{switch.l2}\label{switch.l2}}

\begin{verbatim}
(function switch (l r)
    (`(let (switch:= (,[@fst $l])) (switch:val (,[@frst $l]))
        (,(with return
            {(continuation aux (remaining else-clause)
                (if [emt? $remaining]
                    {return $else-clause}
                    {aux [@rst $remaining]
                        (`(if (,[lst (` or $r) [meta:map [@rst [meta:reverse
                                [@fst $remaining] $r]] $r (function _ (e r)
                                [llllst (` invoke $r) (` $switch:= $r)
                                (` $switch:val $r) $e emt $r]) $r] $r])
                            (,[@fst [meta:reverse [@fst $remaining] $r]])
                                ,$else-clause) $r)}))
                [@rst [meta:reverse [@rrst $l] $r]] [@fst [meta:reverse $l $r]]})))$r))
\end{verbatim}

\hypertarget{test7.l2}{%
\paragraph{test7.l2}\label{test7.l2}}

\begin{verbatim}
(switch = #10
    (#20 [printf (" d is 20!)])
    (#10 [printf (" d is 10!)])
    (#30 [printf (" d is 30!)])
    [printf (" s is something else.)])
\end{verbatim}

\hypertarget{shell-6}{%
\paragraph{shell}\label{shell-6}}

\begin{verbatim}
./bin/l2compile "bin/x86_64.o" abbreviations.l2 comments.l2 dereference.l2 numbers64.l2 \
    backquote.l2 let.l2 boolean.l2 switch.l2 - test7.l2
\end{verbatim}

\hypertarget{characters}{%
\subsubsection{Characters}\label{characters}}

With \texttt{\#} implemented, a somewhat more readable implementation of
characters is possible. The \texttt{char} function takes a singleton
list containing a symbol of one character and returns its ascii encoding
using the \texttt{\#} expression. Its implementation and use follows:

\hypertarget{characters.l2}{%
\paragraph{characters.l2}\label{characters.l2}}

\begin{verbatim}
(function char (l r) (switch char= [@ffst $l]
    (-!- (` #33 $r)) (-"- (` #34 $r)) (-#- (` #35 $r)) (-$- (` #36 $r)) (-%- (` #37 $r))
    (-&- (` #38 $r)) (-'- (` #39 $r)) (-*- (` #42 $r)) (-+- (` #43 $r)) (-,- (` #44 $r))
    (--- (` #45 $r)) (-.- (` #46 $r)) (-/- (` #47 $r)) (-0- (` #48 $r)) (-1- (` #49 $r))
    (-2- (` #50 $r)) (-3- (` #51 $r)) (-4- (` #52 $r)) (-5- (` #53 $r)) (-6- (` #54 $r))
    (-7- (` #55 $r)) (-8- (` #56 $r)) (-9- (` #57 $r)) (-:- (` #58 $r)) (-;- (` #59 $r))
    (-<- (` #60 $r)) (-=- (` #61 $r)) (->- (` #62 $r)) (-?- (` #63 $r)) (-@- (` #64 $r))
    (-A- (` #65 $r)) (-B- (` #66 $r)) (-C- (` #67 $r)) (-D- (` #68 $r)) (-E- (` #69 $r))
    (-F- (` #70 $r)) (-G- (` #71 $r)) (-H- (` #72 $r)) (-I- (` #73 $r)) (-J- (` #74 $r))
    (-K- (` #75 $r)) (-L- (` #76 $r)) (-M- (` #77 $r)) (-N- (` #78 $r)) (-O- (` #79 $r))
    (-P- (` #80 $r)) (-Q- (` #81 $r)) (-R- (` #82 $r)) (-S- (` #83 $r)) (-T- (` #84 $r))
    (-U- (` #85 $r)) (-V- (` #86 $r)) (-W- (` #87 $r)) (-X- (` #88 $r)) (-Y- (` #89 $r))
    (-Z- (` #90 $r)) (-\- (` #92 $r)) (-^- (` #94 $r)) (-_- (` #95 $r)) (-`- (` #96 $r))
    (-a- (` #97 $r)) (-b- (` #98 $r)) (-c- (` #99 $r)) (-d- (` #100 $r)) (-e- (` #101 $r))
    (-f- (` #102 $r)) (-g- (` #103 $r)) (-h- (` #104 $r)) (-i- (` #105 $r))
    (-j- (` #106 $r)) (-k- (` #107 $r)) (-l- (` #108 $r)) (-m- (` #109 $r))
    (-n- (` #110 $r)) (-o- (` #111 $r)) (-p- (` #112 $r)) (-q- (` #113 $r))
    (-r- (` #114 $r)) (-s- (` #115 $r)) (-t- (` #116 $r)) (-u- (` #117 $r))
    (-v- (` #118 $r)) (-w- (` #119 $r)) (-x- (` #120 $r)) (-y- (` #121 $r))
    (-z- (` #122 $r)) (-|- (` #124 $r)) (-~- (` #126 $r)) (` #0 $r)))
\end{verbatim}

\hypertarget{test8.l2}{%
\paragraph{test8.l2}\label{test8.l2}}

\begin{verbatim}
[putchar (char A)]
\end{verbatim}

\hypertarget{shell-7}{%
\paragraph{shell}\label{shell-7}}

\begin{verbatim}
./bin/l2compile "bin/x86_64.o" abbreviations.l2 comments.l2 dereference.l2 numbers64.l2 \
    backquote.l2 let.l2 boolean.l2 switch.l2 characters.l2 - test8.l2
\end{verbatim}

\hypertarget{strings}{%
\subsubsection{Strings}\label{strings}}

The above exposition has purposefully avoided making strings because it
is tedious to do using only literal and reference arithmetic. The quote
function takes a list of symbols and returns the sequence of operations
required to write its ascii encoding into memory. (An extension to this
rule occurs when instead of a symbol, a fragment that is a list of
fragments is encountered. In this case the value of the fragment is
taken as the character to be inserted.) These "operations" are
essentially reserving enough storage for the bytes of the input, putting
the characters into that memory, and returning the address of that
memory. Because the stack-frame of a function is destroyed upon its
return, strings implemented in this way should not be returned. Quote is
implemented below:

\hypertarget{strings.l2}{%
\paragraph{strings.l2}\label{strings.l2}}

\begin{verbatim}
(function " (l r) (with return
    {(continuation add-word (str index instrs)
        (if [emt? $str]
            {return (`(with dquote:return
                (,[llst (` begin $r) [llst (` storage $r) (` dquote:str $r)
                        (with return {(continuation _ (phs num)
                            (if $num
                                {_ [lst (` #0 $r) $phs $r] [- $num #1]}
                                {return $phs})) emt [+[/ $index (unit)]#1]}) $r]
                    [meta:reverse [lst (`{dquote:return dquote:str}$r) $instrs $r]$r]$r]))
                        $r)}
        
        (if (and [emt? [@fst $str]] [emt? [@rst $str]])
            {add-word [@rst $str] [+ $index #1]
                [lst (`[setb [+ dquote:str (,[value->literal $index $r])] #0]$r) $instrs
                    $r]}
                
        (if (and [emt? [@fst $str]] [symbol? [@frst $str]])
            {add-word [@rst $str] [+ $index #1]
                [lst (`[setb [+ dquote:str (,[value->literal $index $r])] #32]$r) $instrs
                    $r]}
        
        (if [emt? [@fst $str]] {add-word [@rst $str] $index $instrs}
                
        (if [symbol? [@fst $str]]
            {add-word [lst [@rfst $str] [@rst $str] $r] [+ $index #1]
                [lst (`[setb [+ dquote:str (,[value->literal $index $r])]
                    (,[char [lst [lst [@ffst $str] emt $r] emt $r]$r emt])]$r) $instrs $r]}
            
            {add-word [@rst $str] [+ $index #1]
                [lst (`[setb [+ dquote:str (,[value->literal $index $r])] (,[@fst $str])]$r)
                    $instrs $r]})))))) $l #0 emt}))
\end{verbatim}

\hypertarget{test9.l2}{%
\paragraph{test9.l2}\label{test9.l2}}

\begin{verbatim}
[printf (" This is how the quote macro is used. (# 10) Now we are on a new line because 10
    is a line feed.)]
\end{verbatim}

\hypertarget{shell-8}{%
\paragraph{shell}\label{shell-8}}

\begin{verbatim}
./bin/l2compile "bin/x86_64.o" abbreviations.l2 comments.l2 dereference.l2 numbers64.l2 \
    backquote.l2 let.l2 boolean.l2 switch.l2 characters.l2 strings.l2 - test9.l2
\end{verbatim}

\hypertarget{closures}{%
\subsubsection{Closures}\label{closures}}

A restricted form of closures can be implemented in L2. The key to their
implementation is to \texttt{jump} out of the function that is supposed
to provide the lexical environment. By doing this instead of merely
returning from the environment function, the stack-pointer and thus the
stack-frame of the environment are preserved. The following example
implements a function that receives a single argument and "returns"
(more accurately: jumps out) a continuation that adds this value to its
own argument. But first, the following transformations are needed:

\begin{verbatim}
(lambda (args ...) expr0)
->
(continuation lambda0 (cont0 args ...)
    {$cont0 expr0})

(; func0 args ...)
->
(with return [func0 return args ...])

(: cont0 args ...)
->
(with return {cont0 return args ...})
\end{verbatim}

These are implemented and used as follows:

\hypertarget{closures.l2}{%
\paragraph{closures.l2}\label{closures.l2}}

\begin{verbatim}
(function lambda (l r)
    (`(continuation lambda0 (,[lst (` cont0 $r) [@fst $l] $r])
        {$cont0 (,[@frst $l])})$r))

(function ; (l r)
    (`(with semicolon:return (,[lllst (` invoke $r) [@fst $l] (` semicolon:return $r)
        [@rst $l] $r]))$r))

(function : (l r)
    (`(with colon:return (,[lllst (` jump $r) [@fst $l] (` colon:return $r) [@rst $l] $r]))
        $r))
\end{verbatim}

\hypertarget{test10.l2}{%
\paragraph{test10.l2}\label{test10.l2}}

\begin{verbatim}
(function adder (cont x)
    {$cont (lambda (y) [+ $x $y])})

(let (add5 (; adder #5)) (add7 (; adder #7))
    (begin
        [printf (" %i,) (: $add5 #2)]
        [printf (" %i,) (: $add7 #3)]
        [printf (" %i,) (: $add5 #1)]))
\end{verbatim}

\hypertarget{shell-9}{%
\paragraph{shell}\label{shell-9}}

\begin{verbatim}
./bin/l2compile "bin/x86_64.o" abbreviations.l2 comments.l2 dereference.l2 numbers64.l2 \
    backquote.l2 let.l2 boolean.l2 switch.l2 characters.l2 strings.l2 closures.l2 - \
    test10.l2
\end{verbatim}

\hypertarget{assume}{%
\subsubsection{Assume}\label{assume}}

There are far fewer subtle ways to trigger undefined behaviors in L2
than in other unsafe languages because L2 does not have dereferencing,
arithmetic operators, types, or other such functionality built in; the
programmer has to implement this functionality themselves in
\href{assets/x86_64.s}{assembly routines callable from L2}. This shift
in responsibility means that any L2 compiler is freed up to treat
invocations of undefined behaviors in L2 code as intentional. The
following usage of undefined behavior within the function
\texttt{assume} is inspired by
\href{https://blog.regehr.org/archives/1096}{Regehr}. The function
\texttt{assume}, which compiles \texttt{y} assuming that the condition
\texttt{x} holds, implements the following transformation.

\begin{verbatim}
(assume x y)
->
(with return
    {(continuation tempas0 ()
        (if x {return y} (begin)))})
\end{verbatim}

This is implemented as follows:

\hypertarget{assume.l2}{%
\paragraph{assume.l2}\label{assume.l2}}

\begin{verbatim}
(function assume (l r)
    (`(with assume:return
        {(continuation assume:tempas0 ()
            (if (,[@fst $l]) {assume:return (,[@frst $l])} (begin)))})$r))
\end{verbatim}

\hypertarget{test11.l2}{%
\paragraph{test11.l2}\label{test11.l2}}

\begin{verbatim}
(function foo (x y)
    (assume [not [= $x $y]] (begin
        [setb $x (char A)]
        [setb $y (char B)]
        [printf (" %c) [getb $x]])))

[foo (" C) (" D)]
\end{verbatim}

In the function \texttt{foo}, if \texttt{\$x} were equal to
\texttt{\$y}, then the else branch of the \texttt{assume}'s \texttt{if}
expression would be taken. Since this branch does nothing,
\texttt{continuation}'s body expression would finish evaluating. But
this is the undefined behavior stated in
\protect\hyperlink{continuation}{the first paragraph of the description
of the \texttt{continuation} expression}. Therefore an L2 compiler does
not have to worry about what happens in the case that \texttt{\$x}
equals \texttt{\$y}. In light of this and the fact that the \texttt{if}
condition is pure, the whole \texttt{assume} expression can be replaced
with the first branch of \texttt{assume}'s \texttt{if} expression. And
more importantly, the the first branch of \texttt{assume}'s \texttt{if}
expression can be optimized assuming that \texttt{\$x} is not equal to
\texttt{\$y}. Therefore, a hypothetical optimizing compiler would also
replace the last \texttt{{[}getb\ \$x{]}}, a load from memory, with
\texttt{(char\ A)}, a constant.

\hypertarget{shell-10}{%
\paragraph{shell}\label{shell-10}}

\begin{verbatim}
./bin/l2compile "bin/x86_64.o" abbreviations.l2 comments.l2 dereference.l2 numbers64.l2 \
    backquote.l2 let.l2 boolean.l2 switch.l2 characters.l2 strings.l2 assume.l2 - \
    test11.l2
\end{verbatim}

Note that the \texttt{assume} expression can also be used to achieve C's
\texttt{restrict} keyword simply by making its condition the conjunction
of inequalities on the memory locations of the extremeties of the
"arrays" in question.

\hypertarget{fields}{%
\subsubsection{Fields}\label{fields}}

L2 has no built-in mechanism for record and union types. The most naive
way to do record types in L2 would be to create a getter function,
setter function, and offset calculation function for every field where
these functions simply access and/or mutate the desired memory
locations. However this solution is untenable because of the amount of
boilerplate that one would have to write. A better solution is to
aggregate the offset, size, getter, and setter of each field into a
higher-order macro that supplies this information into any macro that is
passed to it. This way, generic getter, setter, address-of, offset-of,
and sizeof functions can be defined once and used on any field. More
concretely, the following transformations are what we want:

\begin{verbatim}
(offset-of expr0)
->
(expr0 offset-aux)

(offset-aux expr0 ...)
->
expr0

(size-of expr0)
->
(expr0 size-of-aux)

(size-of-aux expr0 expr1 ...)
->
expr1

(getter-of expr0)
->
(expr0 getter-of-aux)

(getter-of-aux expr0 expr1 expr2 ...)
->
expr2

(setter-of expr0)
->
(expr0 setter-of-aux)

(setter-of-aux expr0 expr1 expr2 expr3 ...)
->
expr3

(& expr0 expr1)
->
(expr0 &-aux expr1)

(&-aux expr0 expr1 expr2 expr3 expr4 ...)
->
[+ expr4 expr0]

(@ expr0 expr1)
->
(expr0 @-aux expr1)

(@-aux expr0 expr1 expr2 expr3 expr4 ...)
->
[expr2 [+ expr4 expr0]]

(setf expr0 expr1 expr2)
->
(expr0 setf-aux expr1 expr2)

(setf-aux expr0 expr1 expr2 expr3 expr4 expr5)
->
[expr3 [+ expr4 expr0] expr5]
\end{verbatim}

Why? Because if we define the macro \texttt{car} by the transformation
\texttt{(car\ expr0\ exprs\ ...)\ -\textgreater{}\ (expr0\ \#0\ \#8\ get8b\ set8b\ exprs\ ...)}
and \texttt{cdr} by the transformation
\texttt{(cdr\ expr0\ exprs\ ...)\ -\textgreater{}\ (expr0\ \#8\ \#8\ get8b\ set8b\ exprs\ ...)},
then we get the following outcomes:

\begin{verbatim}
(offset-of car) -> (car offset-of-aux) -> (offset-of-aux #0 #8 get8b set8b) -> #0
(size-of car) -> (car size-of-aux) -> (size-of-aux #0 #8 get8b set8b) -> #8
(getter-of car) -> (car getter-of-aux) -> (getter-of-aux #0 #8 get8b set8b) -> get8b
(setter-of car) -> (car setter-of-aux) -> (setter-of-aux #0 #8 get8b set8b) -> set8b
(& car expr) -> (car &-aux expr) -> (&-aux #0 #8 get8b set8b expr) -> [+ expr #0]
(@ car expr) -> (car @-aux expr) -> (@-aux #0 #8 get8b set8b expr) -> [get8b [+ expr #0]]
(setf car expr val) -> (car setf-aux expr val) -> (setf-aux #0 #8 get8b set8b expr val) ->
    [set8b [+ expr #0] val]

(offset-of cdr) -> ... -> #8
(size-of cdr) -> ... -> #8
(getter-of cdr) -> ... -> get8b
(setter-of cdr) -> ... -> set8b
(& cdr expr) -> ... -> [+ expr #8]
(@ cdr expr) -> ... -> [get8b [+ expr #8]]
(setf cdr expr val) -> ... -> [set8b [+ expr #8] val]
\end{verbatim}

To recapitulate, we localized and separated out the definition of a
field from the various operations that can be done on it. Since dozens
of fields can potentially be used in a program, it makes sense to define
a helper function, \texttt{mk-field}, that creates them. What follows is
the implementation of this helper function and the aforementioned
transformations:

\hypertarget{fields.l2}{%
\paragraph{fields.l2}\label{fields.l2}}

\begin{verbatim}
(function offset-of (l r) (`((,[@fst $l]) offset-of-aux)$r))

(function offset-of-aux (l r) [@fst $l])

(function size-of (l r) (`((,[@fst $l]) size-of-aux)$r))

(function size-of-aux (l r) [@frst $l])

(function getter-of (l r) (`((,[@fst $l]) getter-of-aux)$r))

(function getter-of-aux (l r) [@frrst $l])

(function setter-of (l r) (`((,[@fst $l]) setter-of-aux)$r))

(function setter-of-aux (l r) [@frrrst $l])

(function & (l r) (`((,[@fst $l]) &-aux (,[@frst $l]))$r))

(function &-aux (l r) (`[+ (,[@frrrrst $l]) (,[@fst $l])]$r))

(function @ (l r) (`((,[@fst $l]) @-aux (,[@frst $l]))$r))

(function @-aux (l r) (`[(,[@frrst $l]) [+ (,[@frrrrst $l]) (,[@fst $l])]]$r))

(function setf (l r) (`((,[@fst $l]) setf-aux (,[@frst $l]) (,[@frrst $l]))$r))

(function setf-aux (l r) (`[(,[@frrrst $l]) [+ (,[@frrrrst $l]) (,[@fst $l])]
    (,[@frrrrrst $l])]$r))

(function mk-field (l r offset size)
    [lllllst [@fst $l] [value->literal $offset $r] [value->literal $size $r]
        (switch = $size (#1 (` get1b $r)) (#2 (` get2b $r)) (#4 (` get4b $r))
            (#8 (` get8b $r)) (`(null)$r))
        (switch = $size (#1 (` set1b $r)) (#2 (` set2b $r)) (#4 (` set4b $r))
            (#8 (` set8b $r)) (`(null)$r))
        [@rst $l] $r])
\end{verbatim}

\hypertarget{somefields.l2}{%
\paragraph{somefields.l2}\label{somefields.l2}}

\begin{verbatim}
(function cons-cell (l r) [mk# $r #16])

(function car (l r) [mk-field $l $r #0 #8])

(function cdr (l r) [mk-field $l $r #8 #8])
\end{verbatim}

\hypertarget{test12.l2}{%
\paragraph{test12.l2}\label{test12.l2}}

\begin{verbatim}
(storage mycons (begin) (begin))
(setf car mycons (char A))
(setf cdr mycons (char a))
[putchar (@ car mycons)]
[putchar (@ cdr mycons)]
\end{verbatim}

\hypertarget{shell-11}{%
\paragraph{shell}\label{shell-11}}

\begin{verbatim}
./bin/l2compile "bin/x86_64.o" abbreviations.l2 comments.l2 dereference.l2 numbers64.l2 \
    backquote.l2 let.l2 boolean.l2 switch.l2 characters.l2 strings.l2 assume.l2 fields.l2 \
    somefields.l2 - test12.l2
\end{verbatim}

Note that there is no struct definition in the code, there are only
definitions of the fields we need to work with. The negative consequence
of this is that we lose C's type safety and portability. The positive
consequences are that we gain control over the struct packing, we are
now able to use the same field definitions across several conceptually
different structs, and that we can overlap our fields in completely
arbitrary ways.

\end{document}
